%%%%%%%%%%%%%%%%%%%%%%%%%%%%%%%%%%%%%%%%%
% Journal Article
% LaTeX Template
% Version 1.3 (9/9/13)
%
% This template has been downloaded from:
% http://www.LaTeXTemplates.com
%
% Original author:
% Frits Wenneker (http://www.howtotex.com)
%
% License:
% CC BY-NC-SA 3.0 (http://creativecommons.org/licenses/by-nc-sa/3.0/)
%
%%%%%%%%%%%%%%%%%%%%%%%%%%%%%%%%%%%%%%%%%

%----------------------------------------------------------------------------------------
%	PACKAGES AND OTHER DOCUMENT CONFIGURATIONS
%----------------------------------------------------------------------------------------

\documentclass[twoside]{article}

\usepackage{amsthm}
\usepackage{amsmath}
\usepackage{verbatim}

\usepackage{lipsum} % Package to generate dummy text throughout this template

\usepackage[sc]{mathpazo} % Use the Palatino font
\usepackage[T1]{fontenc} % Use 8-bit encoding that has 256 glyphs
\linespread{1.05} % Line spacing - Palatino needs more space between lines
\usepackage{microtype} % Slightly tweak font spacing for aesthetics

\usepackage[hmarginratio=1:1,top=32mm,columnsep=20pt]{geometry} % Document margins
\usepackage{multicol} % Used for the two-column layout of the document
\usepackage[hang, small,labelfont=bf,up,textfont=it,up]{caption} % Custom captions under/above floats in tables or figures
\usepackage{booktabs} % Horizontal rules in tables
\usepackage{float} % Required for tables and figures in the multi-column environment - they need to be placed in specific locations with the [H] (e.g. \begin{table}[H])
\usepackage{hyperref} % For hyperlinks in the PDF

\usepackage{lettrine} % The lettrine is the first enlarged letter at the beginning of the text
\usepackage{paralist} % Used for the compactitem environment which makes bullet points with less space between them

\usepackage{abstract} % Allows abstract customization
\renewcommand{\abstractnamefont}{\normalfont\bfseries} % Set the "Abstract" text to bold
\renewcommand{\abstracttextfont}{\normalfont\small\itshape} % Set the abstract itself to small italic text

\usepackage{titlesec} % Allows customization of titles
\renewcommand\thesection{\Roman{section}} % Roman numerals for the sections
\renewcommand\thesubsection{\Roman{subsection}} % Roman numerals for subsections
\titleformat{\section}[block]{\large\scshape\centering}{\thesection.}{1em}{} % Change the look of the section titles
\titleformat{\subsection}[block]{\large}{\thesubsection.}{1em}{} % Change the look of the section titles

\usepackage{fancyhdr} % Headers and footers
\pagestyle{fancy} % All pages have headers and footers
\fancyhead{} % Blank out the default header
\fancyfoot{} % Blank out the default footer
\fancyhead[C]{Running title $\bullet$ November 2012 $\bullet$ Vol. XXI, No. 1} % Custom header text
\fancyfoot[RO,LE]{\thepage} % Custom footer text

\newtheorem{defi}{Definition}
\newtheorem{lemm}{Lemma}

%----------------------------------------------------------------------------------------
%	TITLE SECTION
%----------------------------------------------------------------------------------------

\title{\vspace{-15mm}\fontsize{24pt}{10pt}\selectfont\textbf{Article Title}} % Article title

\author{
\large
\textsc{John Smith}\thanks{A thank you or further information}\\[2mm] % Your name
\normalsize University of California \\ % Your institution
\normalsize \href{mailto:john@smith.com}{john@smith.com} % Your email address
\vspace{-5mm}
}
\date{}

%----------------------------------------------------------------------------------------

\begin{document}

\maketitle % Insert title

\thispagestyle{fancy} % All pages have headers and footers

%----------------------------------------------------------------------------------------
%	ABSTRACT
%----------------------------------------------------------------------------------------

\begin{abstract}

\noindent  % Dummy abstract text
%this is a abstract.

\lipsum[1] % Dummy text

\end{abstract}

%----------------------------------------------------------------------------------------
%	ARTICLE CONTENTS
%----------------------------------------------------------------------------------------

\begin{multicols}{2} % Two-column layout throughout the main article text

%------------------------------------------------

\section{Introduction}

\lettrine[nindent=0em,lines=3]{L} orem ipsum dolor sit amet, consectetur adipiscing elit.
%\input{sections/Sec1Intro}
\lipsum[2-3] % Dummy text

%------------------------------------------------

\section{Related work}
%\input{sections/Sec2RelatedWork}
\lipsum[4] % Dummy text

%------------------------------------------------

\section{Model}

%Maecenas sed ultricies felis. Sed imperdiet dictum arcu a egestas. 
%\begin{compactitem}
%\item Donec dolor arcu, rutrum id molestie in, viverra sed diam
%\item Curabitur feugiat
%\item turpis sed auctor facilisis
%\item arcu eros accumsan lorem, at posuere mi diam sit amet tortor
%\item Fusce fermentum, mi sit amet euismod rutrum
%\item sem lorem molestie diam, iaculis aliquet sapien tortor non nisi
%\item Pellentesque bibendum pretium aliquet
%\end{compactitem}

\subsection{Symbol definition}
This article considers a shared memory system.
The system is modeled as $m$ processors and a task set $\Gamma = \{\tau_0, \tau_1, ..., \tau_{n-1} \}$. \\
Each task $\tau_i$ is characterized by its worst-case execution time $C_i$ and its period $T_i$, both of which are assumed to be integer multiples of a system unit time.
We consider real-time tasks with implicit deadlines.
That is, $T_i$ is also the relative deadline of task $\tau_i$.
The weight for task $\tau_i$ is defined as $W_i = \dfrac{C_i}{T_i}$, and the system utilization is $U_\Gamma = \sum\limits_{i=0}^{n-1} W_i$.
We assume that $0 < W_i < 1$, and $0 < U_\Gamma < m$. \\
A job $J_{i,j}$ is the $j$th task instance of task $\tau_i$. It arrives at time $JStart_{i,j} = j \cdot T_i$ and need to complete its execution by its deadline at time $JEnd_{i,j} = (j+1) \cdot T_i$.
Assuming that the first job of each task arrives at time 0.
That is, $\forall \tau_i \in \Gamma, JStart_{i,0} = 0$.  \\
The multiprocessor real-time scheduling problem is to construct a peroidc schedule for the above tasks, which allocates exactly $C_i$ time units of a processor to task $\tau_i$ within each interval $[(k-1) \cdot T_i, k \cdot T_i)$ for all $k \in \{1,2,3,...\}$, subject to the following constraints:
\begin{compactitem}
\item C1: A processor can only be allocated to one task at any time, that is, processors cannot be shared concurrently;
\item C2: A task can only be allocated at most one processor at any time, that is, tasks are not parallel and thus cannot occupy more than one processor at any time.
\end{compactitem}
The least common multiple of all tasks' period is $H_\Gamma$.
Because of the periodic property of the problem, we only consider the schedule from time 0 to time $H_\Gamma$.
We split the time range $[0, H_\Gamma)$ into $znum$ zones with consecutive deadlines.
Each zone $Z_i$ start at time $ZStart_i$ and end at time $ZEnd_i$.
$\forall c, \exists (i,j), ZStart_c = j \cdot T_i, ZStart_c < ZEnd_c$ and $ZEnd_c = ZStart_{c+1}$($c = 0,...,znum-2$). \\
The scheduling during $[0, H_\Gamma)$ is expressed by a matrix $M$ with $n$ rows and $znum$ columns.
Elements in M denoted as $alloc_{i,j}$ represents the execution time units assigned to $\tau_i$ in $Z_j$.
The scheduling matrix only specifies the assigned units amount, without concerning the exact time and processor a task is to execute.\\
\begin{table}[H]
\caption{Symbol table}
\centering
\label{tab_sym}
\begin{tabular}{ll}
	\hline
	$m$            & number of processors in system \\
	$n$            & number of tasks in system \\
	$znum$         & number of zones in system \\
	$\tau_i$       & the $i$th task \\
	$C_i$          & worst case execution time of $\tau_i$ \\
	$T_i$          & period of $\tau_i$ \\
	$W_i$          & weight of $\tau_i$ \\
	$\Gamma$       & task set, \{$\tau_0$, ..., $\tau_{n-1}$\} \\
	$H_\Gamma$     & hyperperiod of $\Gamma$ \\
	$U_\Gamma$     & the system utilization \\
	$J_{i,j}$      & the $j$th job of $\tau_i$ \\
	$JStart_{i,j}$ & the start time of $J_{i,j}$ \\
	$JEnd_{i,j}$   & the end time of $J_{i,j}$ \\
	$Z_i$					 & the $i$th Zone\\
	$ZStart_i$     & the start time of $Z_i$ \\
	$ZEnd_i$       & the end time of $Z_i$ \\
	$ZWidth_i$     & the width of $Z_i$ \\
	$Rem_i$        & unassigned time units in $Z_i$ \\
	$L_i$			     & laxity of $\tau_i$ \\
	$R_i$					 & remaining execution time of $\tau_i$ \\
	$alloc_{i,j}$  & assigned time units for $\tau_i$ in $Z_j$ \\
	\hline
\end{tabular}
\end{table}
All the symbols are listed in \ref{tab_sym}.

\subsection{Overload effect}
\begin{defi}
The minimum load of $\tau_i$ in $Z_k$ is the minimum units of execution time that must be assigned to $\tau_i$ in $Z_k$ to meet the deadline.
\[
	Lt_{i,k}=MAX(0,ZWidth_k-L_{i,k})
\]
\end{defi}

\begin{defi}
The maximum load of $\tau_i$ in $Z_k$ is the maximum units of execution time that can be assigned to $\tau_i$ in $Z_k$ considering the WCET of a job and the width of a Zone(to ensure a task can only be executed in a processor at a time).
\[
	Mt_{i,k}=MIN(R_{i,k}, ZWidth_k)
\]
\end{defi}

\begin{defi}
The execution assignment is \textbf{rational} in $Z_k$ if execution time assigned to each task is between minimum load and maximum load, and the sum of execution time of tasks in $\Gamma$ doesn't exceed the capacity of $Z_k$.
\[
	\forall i \in [0,n) : Lt_{i,k} \leq alloc_{i,k} \leq Mt_{i,k}
\]
\[
	\sum\limits_{i=0}^{n-1}alloc_{i,k} \leq m \cdot ZWidth_k
\]
\end{defi}

\begin{defi}
\textbf{Overload effect} occurs if the total minimum load of tasks in $\Gamma$ is more than the capacity of $Z_k$.
\[
	\sum\limits_{i=0}^{n-1}Lt_{i,k} > m \cdot ZWidth_k
\]
\end{defi}


%\lipsum[5] % Dummy text

%------------------------------------------------

\section{Algorithm}

%\begin{table}[H]
%\caption{Example table}
%\centering
%\begin{tabular}{llr}
%\toprule
%\multicolumn{2}{c}{Name} \\
%\cmidrule(r){1-2}
%First name & Last Name & Grade \\
%\midrule
%John & Doe & $7.5$ \\
%Richard & Miles & $2$ \\
%\bottomrule
%\end{tabular}
%\end{table}

\lipsum[6] % Dummy text

%\begin{equation}
%\label{eq:emc}
%e = mc^2
%\end{equation}

\lipsum[7] % Dummy text

%------------------------------------------------

\section{Correctness of algorithm}
\lipsum[10] % Dummy text

%------------------------------------------------

\section{Assessment}

\subsection{Complexity}

\lipsum[8] % Dummy text

\subsection{Simulation}

\lipsum[9] % Dummy text

%----------------------------------------------------------------------------------------
%	REFERENCE LIST
%----------------------------------------------------------------------------------------

\begin{thebibliography}{99} % Bibliography - this is intentionally simple in this template

\bibitem[Figueredo and Wolf, 2009]{Figueredo:2009dg}
Figueredo, A.~J. and Wolf, P. S.~A. (2009).
\newblock Assortative pairing and life history strategy - a cross-cultural
  study.
\newblock {\em Human Nature}, 20:317--330.
 
\end{thebibliography}

%----------------------------------------------------------------------------------------

\end{multicols}

\end{document}
